\documentclass[aspectratio=169]{beamer}

% select theme
\usetheme{CambridgeUS}
\usefonttheme{professionalfonts}
\usecolortheme{beaver}
\useinnertheme{rectangles}
\setbeamertemplate{theorems}[ams style] 

\usepackage{kotex}
\usepackage{fancyvrb}
\usepackage{color}
\usepackage{graphicx}
\usepackage[ruled,vlined]{algorithm2e}
\usepackage{hyperref}

% \usepackage{amsmath}

% \setfont


% \newcommand{\mathbb{E}}{\mathbb{E}}



\begin{document}



% title slide
\begin{frame}
	\title{MINE: Mutual Information Neural Estimation}
	\subtitle{ICML 2018}
	\author{Jaehyun Ko}
	\date{\today}
	\titlepage
\end{frame}


% outline slide
\section*{Outline}
\begin{frame}
\tableofcontents
\end{frame}



\section{Introduction}
\begin{frame}
	\frametitle{Introduction}
	\textbf{Motivation} \\ Estimate mutual information between two random variables using neural networks. \\
\end{frame}

\begin{frame}
	\textbf{Toy Data} \\ Two gaussian distribution
	\begin{align*}
		X &\sim \mathcal{N}(0,
		\begin{bmatrix}
			1 & 0 \\
			0 & 1
		\end{bmatrix})\\
		Y &\sim \mathcal{N}(0, 
		\begin{bmatrix}
			1 & 0.5 \\
			0.5 & 1
		\end{bmatrix})
	\end{align*}
	We can calculate mutual information of these two random variables as follows.
\end{frame}

\section{Backgrounds}
\subsection{Information Theory}
\begin{frame}
	\frametitle{Entropy}
	\textbf{Entropy} is a measure of the uncertainty of a random variable.
	\begin{definition}
		\textbf{Entropy} For any probability density fuction p, entropy is defied as
		\begin{equation*}H(x) = \mathbb{E}_p [-\log p(x)] - \int {p(x)\log{p(x)}dx}\end{equation*}
	\end{definition}
	\begin{itemize}
		\item \textbf{Entropy} is a measure of the uncertainty of a random variable.
		\item average bit-length to representate RV \cite{shannon1948mathematical}.
	\end{itemize}
\end{frame}

\begin{frame}
	\frametitle{Cross Entropy}
	The cross-entropy between two probability distributions $p$ and $q$
	over the same underlying set of events measures
	the average number of bits needed to identify an event
	drawn from the set if a coding scheme used for the set
	is optimized for an estimated probability distribution $q$, rather than the true distribution $p$.
	\begin{definition}
		\textbf{Cross Entropy}(CE) is defined as 
		\begin{equation*}H(p, q) = \mathbb{E}_p [-\log q(x)] = \int p(x) \log q(x)dx\end{equation*}
	\end{definition}
	\end{frame}

\begin{frame}
	\frametitle{Kullback-Leibler Divergence}
	\begin{definition}
		\textbf{Kullback-Leibler Divergence} (KLD) For two probability densities p(x), q(x) is defined as
		\begin{equation*}D(p(x)||q(x))
			= \int p(x) \log \frac{p(x)}{q(x)}dx,\end{equation*}
	\end{definition}
	it can be interpreted as difference of two entropy.
	\begin{align*}
		D(p(x)||q(x)) = \int p(x) (-\log{q(x)})dx - \int p(x) (-\log{p(x)})dx \\
		= H(p, q) - H(p)
	\end{align*}
\end{frame}

\begin{frame}
	\frametitle{Mutual Information} 
	MI is a measure of the dependence between two random variables.
	\begin{definition} \label{def:mutual_information}
		\textbf{Mutual Information} (MI) Let $X$ and $Y$ be two random variables with a joint
		distribution $P(x, y)$ and $P_x$, $P_y$ are marginal probability distribution each.
		The Mutual Information $I(X;Y)$ is defined as
		\begin{equation*}I(X;Y) = \mathbb{E}_{P_{xy}} [\log\frac{P_{xy}}{P_x P_y}]\end{equation*}
	\end{definition}	
\end{frame}

\begin{frame}
	\frametitle{Mutual Information(cont.)}
	we can rewrite the mutual information as follows.
	\begin{align*}
		I(X;Z)&= \mathbb{E}_P[-\log P_x] - \mathbb{E}_P[-\log\frac{P_y}{P_{xy}}]\\
		&= H(X)-H(X|Z)
	\end{align*}
	MI between X and Z can be understood as the decrease of the uncertainty in X given Z.
	and it also representated as KLD between joint distribution and product of marginal distribution.
	\begin{equation}\label{eq:kl_form} I(X;Z) = D(P_{xy}||P_{x} \otimes P_{y}) \end{equation}
\end{frame}

\subsection{Donsker-Varadhan Variational Formula}
	\begin{frame}
		\frametitle{Donsker-Varadhan Representation}
		\begin{theorem}
			\textbf{Donsker-Varadhan Representation} (DV)
			Let $X$ be a random variable with domain $\mathcal{X}$, let P, Q be two probability density functions
			and $T$ be a function on $\mathcal{X}$, 
			Then, for any $x \in \mathcal{X}$, the KLD admits the following dual Representation
			\begin{equation*}
				D(P || Q) = \sup_{T:\mathcal{X} \rightarrow \mathbb{R}} \{ \mathbb{E}_P[T] - \log \mathbb{E}_Q[e^{T}]\}
			\end{equation*}
		\end{theorem}
		the proof of theorem consists of two steps.
		\begin{itemize}
			\item \textbf{Step 1} : Existence of supremum in Donsker-Varadhan variational representation
			\item \textbf{Step 2} : Lower bound for the Kullback Liebler Divergence
		\end{itemize}	
	\end{frame}

	\begin{frame}
		\frametitle{Donsker-Varadhan Representation(cont.)}
		\framesubtitle{Existence of supremum in Donsker-Varadhan variational representation}
		\begin{lemma}
			There exists a function $T^*: X \rightarrow \mathbb{R}$ such that satisfies the condition of equality.
		\end{lemma}
		choise $T^* = \log\frac{P}{Q}$, then prove in the following page.
	\end{frame}


	\begin{frame}
		\frametitle{Donsker-Varadhan Representation(cont.)}
		\framesubtitle{Existence of supremum in Donsker-Varadhan variational representation}
		\begin{align}
			D_{\text{KL}}(P|Q) &= \mathbb{E}_P[T^*(X)] - \log(\mathbb{E}_Q[e^{T^*(X)}])\\			
			% \mathbb{E}_P[T(X)] - \log(\mathbb{E}_Q[e^{T(X)}])
			&= \mathbb{E}_P [\log \frac{P(X)}{Q(X)}]-\log(\mathbb{E}_Q[e^{\log\frac{P(X)}{Q(X)}}])\\
			&= D_{\text{KL}}(P|Q) - \log(\mathbb{E}_Q[\frac{P(X)}{Q(X)}])\\
			&= D_{\text{KL}}(P|Q) - \log(\int_{\mathcal{X}} Q(x)\frac{P(x)}{Q(x)}dx)\\
			&= D_{\text{KL}}(P|Q) - \log(\int_{\mathcal{X}} P(x)dx)\\
			&= D_{\text{KL}}(P|Q) - \log(1)\\
			&= D_{\text{KL}}(P|Q)
		\end{align}
	\end{frame}

	\begin{frame}
		\frametitle{Donsker-Varadhan Representation(cont.)}
		\framesubtitle{Lower bound for the Kullback Liebler Divergence}
		\begin{lemma}\label{lemma:lower_bound}
			For any function $T:X \rightarrow \mathbb{R}$ the following inequality holds:
			$D_{\text{KL}}(P | Q) \geq \sup_{T: \mathcal{X} \rightarrow \mathbb{R}} {\mathbb{E}_P[T(X)] - \log \mathbb{E}_Q[e^{T(X)}]}$
		\end{lemma}
		suppose new probability density function $G$ is defined as follows:
		\begin{align}
			G(x) &=\frac{Q(x)e^T}{\mathbb{E}_Q[e^{T(X)}]}\\
			\int_{\mathcal{X}} G(x)dx = \frac{\int_{\mathcal{X}} Q(x)e^T}{\mathbb{E}_Q[e^{T(X)}]} 
			&=\frac{\mathbb{E}_Q[e^{T(X)}]}{\mathbb{E}_Q[e^{T(X)}]} = 1 
		\end{align}
	\end{frame}

	\begin{frame}
		\frametitle{Donsker-Varadhan Representation(cont.)}
		\framesubtitle{Lower bound for the Kullback Liebler Divergence}
		\begin{align}
			&D_{\text{KL}}(P | Q) -\sup_{T: \mathcal{X} \rightarrow \mathbb{R}} {\mathbb{E}_P[T(X)] + \log \mathbb{E}_Q[e^{T(X)}]}\\
			&= \mathbb{E}_P[\log \frac{P(X)}{Q(X)}-T(X)] + \log(\mathbb{E}_Q[e^{T(X)}])\\
			&= \mathbb{E}_P[\log \frac{P(X)}{Q(X) e^{T(X)}}] - \log(\mathbb{E}_Q[e^{T(X)}])\\
			&= \mathbb{E}_P[\log \frac{P(X)\mathbb{E}_Q[e^{T(X)}]}{Q(X) e^{T(X)}}]\\
			&= \mathbb{E}_P[\log \frac{P(X)}{G(X)}]\\
			&= D_{\text{KL}}(P|G) \geq 0
		\end{align}
	\end{frame}
	% \begin{align*}

	% \end{align*} 
	
	% \begin{equation*}

	% \end{equation*}

% \end{frame}

\section{MINE}
\begin{frame}
	\frametitle{MINE}
	\framesubtitle{Mutual Information Neural Estimation}
	in this section, we will Donsker-Varadhan variational formulation in order to estimate mutual information,
	via approximating $T$ using neural network. according to discussion so,
	we can estimate the mutual information by maximizing the following cost function:
	% \autoref{eq:kl_form} 
	\begin{equation}
		I(X;Y) = \sup_{T: \mathcal{X} \times \mathcal{Y} \rightarrow \mathbb{R}}
		{\mathbb{E}_{P_{XY}}[T(X,Y)] - \log \mathbb{E}_{P_X \otimes P_Y}[e^{T(X,Y)}]}
	\end{equation}
\end{frame}






\begin{frame}
	\begin{center}
		\begin{minipage}{0.8\linewidth}
			% \setframetitle{MINE Algorithm}
			\begin{algorithm}[H]
				\SetAlgoLined
				\KwIn{Joint distribution $P_{XY}$ and neural network architecture}
				\KwOut{An estimate of the mutual information $I(X;Y)$}
				Initialize network parameters $\theta$\
				\Repeat{convergence}{
				Draw mini-batch of samples: $(X_1, Y_1), (X_2, Y_2), \ldots, (X_m, Y_m) \sim P_{XY}$\;
				Draw $n$ samples from the marginal distribution: $Y_{1}, Y_{2}, \ldots, Y_{n} \sim P_Y$\;
				Evaluate: $\hat{I}_\theta(X; Y) \rightarrow \frac{1}{n} \sum_{i=1}^n T_\theta(X_i, Y_i) - \log(\frac{1}{n}\sum_{i=1}^n e^{T_\theta(X_i, \tilde{Y_i})})$\;
				Update network parameters: $\theta \rightarrow \theta + \nabla_\theta \hat{I}_\theta(X; Y)$\;
				}
				\Return An estimate of the mutual information $I(X;Y)$\
				\caption{Mutual Information Neural Estimation (MINE)}
				\label{alg:mine}
			\end{algorithm}
		\end{minipage}
	\end{center}
\end{frame}

% \section{Experiments}

% \section{Application}

% \section{Conclusion}

\section{Refrences}
\begin{frame}
	\bibliographystyle{plain} % We choose the "plain" reference style
	\bibliography{ref.bib} % Entries are in the refs.bib file
	% \printbibliography 
\end{frame}

\end{document}

